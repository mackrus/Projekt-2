% Options for packages loaded elsewhere
\PassOptionsToPackage{unicode}{hyperref}
\PassOptionsToPackage{hyphens}{url}
%
\documentclass[
]{article}
\usepackage{amsmath,amssymb}
\usepackage{lmodern}
\usepackage{iftex}
\ifPDFTeX
  \usepackage[T1]{fontenc}
  \usepackage[utf8]{inputenc}
  \usepackage{textcomp} % provide euro and other symbols
\else % if luatex or xetex
  \usepackage{unicode-math}
  \defaultfontfeatures{Scale=MatchLowercase}
  \defaultfontfeatures[\rmfamily]{Ligatures=TeX,Scale=1}
\fi
% Use upquote if available, for straight quotes in verbatim environments
\IfFileExists{upquote.sty}{\usepackage{upquote}}{}
\IfFileExists{microtype.sty}{% use microtype if available
  \usepackage[]{microtype}
  \UseMicrotypeSet[protrusion]{basicmath} % disable protrusion for tt fonts
}{}
\makeatletter
\@ifundefined{KOMAClassName}{% if non-KOMA class
  \IfFileExists{parskip.sty}{%
    \usepackage{parskip}
  }{% else
    \setlength{\parindent}{0pt}
    \setlength{\parskip}{6pt plus 2pt minus 1pt}}
}{% if KOMA class
  \KOMAoptions{parskip=half}}
\makeatother
\usepackage{xcolor}
\usepackage{longtable,booktabs,array}
\usepackage{multirow}
\usepackage{calc} % for calculating minipage widths
% Correct order of tables after \paragraph or \subparagraph
\usepackage{etoolbox}
\makeatletter
\patchcmd\longtable{\par}{\if@noskipsec\mbox{}\fi\par}{}{}
\makeatother
% Allow footnotes in longtable head/foot
\IfFileExists{footnotehyper.sty}{\usepackage{footnotehyper}}{\usepackage{footnote}}
\makesavenoteenv{longtable}
\usepackage{graphicx}
\makeatletter
\def\maxwidth{\ifdim\Gin@nat@width>\linewidth\linewidth\else\Gin@nat@width\fi}
\def\maxheight{\ifdim\Gin@nat@height>\textheight\textheight\else\Gin@nat@height\fi}
\makeatother
% Scale images if necessary, so that they will not overflow the page
% margins by default, and it is still possible to overwrite the defaults
% using explicit options in \includegraphics[width, height, ...]{}
\setkeys{Gin}{width=\maxwidth,height=\maxheight,keepaspectratio}
% Set default figure placement to htbp
\makeatletter
\def\fps@figure{htbp}
\makeatother
\usepackage[normalem]{ulem}
\setlength{\emergencystretch}{3em} % prevent overfull lines
\providecommand{\tightlist}{%
  \setlength{\itemsep}{0pt}\setlength{\parskip}{0pt}}
\setcounter{secnumdepth}{-\maxdimen} % remove section numbering
\ifLuaTeX
  \usepackage{selnolig}  % disable illegal ligatures
\fi
\IfFileExists{bookmark.sty}{\usepackage{bookmark}}{\usepackage{hyperref}}
\IfFileExists{xurl.sty}{\usepackage{xurl}}{} % add URL line breaks if available
\urlstyle{same} % disable monospaced font for URLs
\hypersetup{
  hidelinks,
  pdfcreator={LaTeX via pandoc}}

\author{}
\date{}

\begin{document}

\includegraphics[width=4.46528in,height=2.59867in]{vertopal_35c4d4bda5e64349bff9adbb02318ace/media/image4.png}\includegraphics[width=4.44444in,height=2.58333in]{vertopal_35c4d4bda5e64349bff9adbb02318ace/media/image5.png}

\begin{longtable}[]{@{}
  >{\raggedright\arraybackslash}p{(\columnwidth - 6\tabcolsep) * \real{0.2500}}
  >{\raggedright\arraybackslash}p{(\columnwidth - 6\tabcolsep) * \real{0.2500}}
  >{\raggedright\arraybackslash}p{(\columnwidth - 6\tabcolsep) * \real{0.2500}}
  >{\raggedright\arraybackslash}p{(\columnwidth - 6\tabcolsep) * \real{0.2500}}@{}}
\toprule()
\begin{minipage}[b]{\linewidth}\raggedright
\begin{quote}
\includegraphics[width=0.63889in,height=0.63889in]{vertopal_35c4d4bda5e64349bff9adbb02318ace/media/image1.png}
\end{quote}
\end{minipage} & \begin{minipage}[b]{\linewidth}\raggedright
30 april 2024
\end{minipage} & \begin{minipage}[b]{\linewidth}\raggedright
\begin{quote}
Projekt 2
\end{quote}
\end{minipage} & \begin{minipage}[b]{\linewidth}\raggedright
1 (4)
\end{minipage} \\
\midrule()
\endhead
\bottomrule()
\end{longtable}

\begin{quote}
\includegraphics[width=0.79167in,height=0.15278in]{vertopal_35c4d4bda5e64349bff9adbb02318ace/media/image2.png}
\end{quote}

\includegraphics[width=1.16667in,height=0.13889in]{vertopal_35c4d4bda5e64349bff9adbb02318ace/media/image3.png}

\begin{quote}
\textbf{Institionen för}\\
\textbf{informationsteknologi}\\
Beräkningsvetenskap

Besöksadress:\\
Lägerhyddsvägen 1, hus 10
\end{quote}

Introduktion till Beräkningsvetenskap

\begin{longtable}[]{@{}
  >{\raggedright\arraybackslash}p{(\columnwidth - 2\tabcolsep) * \real{0.5000}}
  >{\raggedright\arraybackslash}p{(\columnwidth - 2\tabcolsep) * \real{0.5000}}@{}}
\toprule()
\begin{minipage}[b]{\linewidth}\raggedright
\begin{quote}
Postadress:\\
Box 337\\
751 05 Uppsala

Telefon:\\
018--471 0000 (växel)

Telefax:\\
018--51 19 25

Hemsida:\\
http://www.it.uu.se
\end{quote}\strut
\end{minipage} & \begin{minipage}[b]{\linewidth}\raggedright
\begin{quote}
\textbf{Projekt 2: Kaotisk pendel}

\emph{I detta projekt ska ni använda resultat från det första projektet
där ni tog fram numeriska lösningar till en ODE, som beskrev en dämpad
och driven pendel. Vi ska nu ta reda på hur långt spetsen på pendeln
rört sig under rörelsen, genom att utnyttja numeriska metoder för
integrering. Du kommer att använda ett par av Pythons (SciPy) egna
lösare för numerisk integrering, quad och trapezoid. Du kommer analysera
trunkeringsfel och noggrannhet. Nyckelbegrepp är \textbf{konvergens}.}
\end{quote}
\end{minipage} \\
\midrule()
\endhead
\begin{minipage}[t]{\linewidth}\raggedright
\begin{quote}
\textbf{Department of}\\
\textbf{Information Technology} Scientific Computing

Visiting address:
\end{quote}\strut
\end{minipage} &
\multirow{2}{*}{\begin{minipage}[t]{\linewidth}\raggedright
\begin{quote}
\textbf{Matematisk beskrivning av en pendel}\\
Om vi endast tar hänsyn till gravitationskraften beskrivs pendelns
rörelse av \emph{𝜃}′′= −\emph{𝑔}har massan \emph{𝑚}och dess längd är
\emph{𝑟}. För att påbörja pendelrörelsen startar vi\emph{𝑅}sin
(\emph{𝜃}), där \emph{𝜃}är vinkeln som pendeln har (se Figur 1). Pendeln
pendeln (vid tiden \emph{𝑡}= 0) från vila (dvs \emph{𝜃}′= 0) från en
given vinkel \emph{𝜃}= \emph{𝜃}0.\emph{𝜃}′= 0 och \emph{𝜃}= \emph{𝜃}0 är
våra begynnelsevillkor, dvs pendelns position och hastighet vid tiden
\emph{𝑡}= 0.
\end{quote}\strut
\end{minipage}} \\
\begin{minipage}[t]{\linewidth}\raggedright
\begin{quote}
Lägerhyddsvägen 1, hus 10

Postal address:\\
Box 337\\
SE-751 05 Uppsala SWEDEN
\end{quote}\strut
\end{minipage} \\
\bottomrule()
\end{longtable}

\begin{quote}
Telephone:\\
+46 18--471 0000 (switch)

Telefax:\\
+46 18--51 19 25
\end{quote}

\begin{longtable}[]{@{}
  >{\raggedright\arraybackslash}p{(\columnwidth - 4\tabcolsep) * \real{0.3333}}
  >{\raggedright\arraybackslash}p{(\columnwidth - 4\tabcolsep) * \real{0.3333}}
  >{\raggedright\arraybackslash}p{(\columnwidth - 4\tabcolsep) * \real{0.3333}}@{}}
\toprule()
\begin{minipage}[b]{\linewidth}\raggedright
\begin{quote}
Web page:
\end{quote}
\end{minipage} &
\multirow{2}{*}{\begin{minipage}[b]{\linewidth}\raggedright
\textbf{Θ}
\end{minipage}} &
\multirow{2}{*}{\begin{minipage}[b]{\linewidth}\raggedright
\textbf{r}
\end{minipage}} \\
\begin{minipage}[b]{\linewidth}\raggedright
\begin{quote}
http://www.it.uu.se
\end{quote}
\end{minipage} \\
\midrule()
\endhead
\bottomrule()
\end{longtable}

\textbf{m}

Figur 1: Illustration av en pendel med massa \emph{𝑚}och längd \emph{𝑟}.

\begin{quote}
I verkligheten måste vi även ta hänsyn till friktionskraft, som vi här
antarär proportionell mot pendels hastighet, \emph{𝑏𝜃}′, där \emph{𝑏}är
en friktionsparameter. Till sist lägger vi på en drivande kraft
\emph{𝐹}0 cos \emph{𝜔𝑡}, där \emph{𝐹}0 är amplituden och
\end{quote}

\begin{longtable}[]{@{}
  >{\raggedright\arraybackslash}p{(\columnwidth - 4\tabcolsep) * \real{0.3333}}
  >{\raggedright\arraybackslash}p{(\columnwidth - 4\tabcolsep) * \real{0.3333}}
  >{\raggedright\arraybackslash}p{(\columnwidth - 4\tabcolsep) * \real{0.3333}}@{}}
\toprule()
\begin{minipage}[b]{\linewidth}\raggedright
\end{minipage} &
\multicolumn{2}{>{\raggedright\arraybackslash}p{(\columnwidth - 4\tabcolsep) * \real{0.6667} + 2\tabcolsep}@{}}{%
\begin{minipage}[b]{\linewidth}\raggedright
2 (4)
\end{minipage}} \\
\midrule()
\endhead
\multirow{5}{*}{\begin{minipage}[t]{\linewidth}\raggedright
\begin{quote}
\includegraphics[width=0.63889in,height=0.63889in]{vertopal_35c4d4bda5e64349bff9adbb02318ace/media/image6.png}

\includegraphics[width=0.80556in,height=0.15278in]{vertopal_35c4d4bda5e64349bff9adbb02318ace/media/image7.png}
\end{quote}

\includegraphics[width=1.16667in,height=0.13889in]{vertopal_35c4d4bda5e64349bff9adbb02318ace/media/image8.png}
\end{minipage}} &
\multicolumn{2}{>{\raggedright\arraybackslash}p{(\columnwidth - 4\tabcolsep) * \real{0.6667} + 2\tabcolsep}@{}}{%
\begin{minipage}[t]{\linewidth}\raggedright
\begin{quote}
\emph{𝜔}frekvensen hos den drivande kraften. Den fullständiga
beskrivningen av pendelrörelsen ges av följande ODE,
\end{quote}
\end{minipage}} \\
& \emph{𝜃}′′= −\emph{𝑔𝑟}sin (\emph{𝜃}) −\emph{𝑏𝑚𝜃}′ +
\emph{𝐹}0\emph{𝑚𝑟}cos (\emph{𝜔𝑡}) \emph{.} & (1) \\
&
\multicolumn{2}{>{\raggedright\arraybackslash}p{(\columnwidth - 4\tabcolsep) * \real{0.6667} + 2\tabcolsep}@{}}{%
\begin{minipage}[t]{\linewidth}\raggedright
\begin{quote}
I ekvation (1) har vi 6 parametrar: \emph{𝑚}, \emph{𝑟}, \emph{𝑔},
\emph{𝑏}, \emph{𝐹}0 och \emph{𝜔}. Genom att införa 2\emph{𝛽}= \emph{𝑚},
\emph{𝜔}2 0= \emph{𝑟}och \emph{𝛾}= \emph{𝑚𝑟𝜔}2 \emph{𝐹}0 0, kan vi
skriva om (1) med endast 4 parametrar (\emph{𝜔}0, \emph{𝜔}, \emph{𝛽}och
\emph{𝛾}),
\end{quote}
\end{minipage}} \\
& \emph{𝜃}′′= −\emph{𝜔}2 0sin (\emph{𝜃}) −2 \emph{𝛽𝜃}′ + \emph{𝛾𝜔}2 0cos
(\emph{𝜔𝑡}) \emph{.} & (2) \\
&
\multicolumn{2}{>{\raggedright\arraybackslash}p{(\columnwidth - 4\tabcolsep) * \real{0.6667} + 2\tabcolsep}@{}}{%
\begin{minipage}[t]{\linewidth}\raggedright
För små pendelrörelser, dvs \textbar{}\emph{𝜃}\textbar{}
\emph{\textless\textless{}} 1, kan vi linearisera (2) genom att

Taylor-utveckla kring sin (0), vilket ger oss sin (\emph{𝜃}) ≃\emph{𝜃}då
\textbar{}\emph{𝜃}\textbar{} \emph{\textless\textless{}} 1. Om vi
samtidigt sätter friktionen till noll, dvs \emph{𝑏}= 0, får vi följande
förenklade

\begin{quote}
(linjära) ODE,

\emph{𝜃}′′= −\emph{𝜔}2 0\emph{𝜃}+ \emph{𝛾𝜔}2 0cos (\emph{𝜔𝑡}) \emph{.}
(3)
\end{quote}

Parametern \emph{𝜔}0 brukar benämnas systemets naturliga (eller egen)
frekvens.
\end{minipage}} \\
\bottomrule()
\end{longtable}

\textbf{Uppgifter}

\begin{quote}
Här följer 3 obligatoriska uppgifter (Deluppgift 1-3), och en slutlig
frivillig uppgift. Dessa uppgifter testar centrala delar av kursen, och
är en bra övning inför den slutliga examinationen.
\end{quote}

\textbf{Deluppgift 1}

\begin{quote}
Vi inleder med den linjära modellen (3), där begynnelsevillkoren sätts
till\emph{𝜃}= \emph{𝜃}′= 0. Man kan visa att vinkelhastigheten ges av,
\end{quote}

\begin{longtable}[]{@{}
  >{\raggedright\arraybackslash}p{(\columnwidth - 8\tabcolsep) * \real{0.2000}}
  >{\raggedright\arraybackslash}p{(\columnwidth - 8\tabcolsep) * \real{0.2000}}
  >{\raggedright\arraybackslash}p{(\columnwidth - 8\tabcolsep) * \real{0.2000}}
  >{\raggedright\arraybackslash}p{(\columnwidth - 8\tabcolsep) * \real{0.2000}}
  >{\raggedright\arraybackslash}p{(\columnwidth - 8\tabcolsep) * \real{0.2000}}@{}}
\toprule()
\begin{minipage}[b]{\linewidth}\raggedright
\emph{𝜃}′(\emph{𝑡}) =
\end{minipage} &
\multicolumn{3}{>{\raggedright\arraybackslash}p{(\columnwidth - 8\tabcolsep) * \real{0.6000} + 4\tabcolsep}}{%
\begin{minipage}[b]{\linewidth}\raggedright
\begin{quote}
\emph{𝛾𝜔}2 0

\emph{𝜔}2 0−\emph{𝜔}2 (\emph{𝜔}0 sin (\emph{𝜔}0 \emph{𝑡}) −\emph{𝜔}sin
(\emph{𝜔𝑡})) \emph{,}
\end{quote}
\end{minipage}} & \begin{minipage}[b]{\linewidth}\raggedright
(4)
\end{minipage} \\
\midrule()
\endhead
\multicolumn{5}{@{}>{\raggedright\arraybackslash}p{(\columnwidth - 8\tabcolsep) * \real{1.0000} + 8\tabcolsep}@{}}{%
\begin{minipage}[t]{\linewidth}\raggedright
\begin{quote}
då \emph{𝜔}≠\emph{𝜔}0. Hastigheten som spetsen rör sig med ges av
\emph{𝑣}(\emph{𝑡}) = \emph{𝑟}· \emph{𝜃}′(\emph{𝑡}). För enkelhetens
skull antar vi att \emph{𝑟}= 1 \emph{𝑚}. Sträckan som pendeln har rört
\end{quote}
\end{minipage}} \\
\multicolumn{2}{@{}>{\raggedright\arraybackslash}p{(\columnwidth - 8\tabcolsep) * \real{0.4000} + 2\tabcolsep}}{%
sig mellan \emph{𝑡}= 0 och \emph{𝑡}= \emph{𝑇}ges av,} &
\multirow{2}{*}{∫\emph{𝑇}} &
\multirow{2}{*}{\begin{minipage}[t]{\linewidth}\raggedright
\begin{quote}
\textbar{}\emph{𝑣}(\emph{𝑡})\textbar{} \emph{𝑑𝑡}
\end{quote}
\end{minipage}} & \multirow{2}{*}{(5)} \\
\multicolumn{2}{@{}>{\raggedright\arraybackslash}p{(\columnwidth - 8\tabcolsep) * \real{0.4000} + 2\tabcolsep}}{%
\emph{𝑆}(\emph{𝑇}) =} \\
\multicolumn{5}{@{}>{\raggedright\arraybackslash}p{(\columnwidth - 8\tabcolsep) * \real{1.0000} + 8\tabcolsep}@{}}{%
\begin{minipage}[t]{\linewidth}\raggedright
\begin{quote}
Skapa en funktion i Python, som givet en tidsvektor returnerar beloppet
av hastigheten (\textbar{}\emph{𝑣}(\emph{𝑡})\textbar) i motsvarande
punkter där den analytiska vinkelhastig-heten ges av (4) . Övriga 3
parametervärden (\emph{𝜔}0, \emph{𝜔}och \emph{𝛾}) ska tilldelas utanför
funktionen, och ska vara inparametrar till funktionen. Redovisa en figur
för hastigheten \textbar{}\emph{𝑣}(\emph{𝑡})\textbar{} mellan \emph{𝑡}=
0 och \emph{𝑡}= 10 då \emph{𝜔}= 2 \emph{𝜋}, \emph{𝜔}0 = 3 \emph{𝜋}och
\emph{𝛾}= 0\emph{.}1. Använd \emph{𝑁}= 4000 intervall för en
kontinuerlig kurva. Skriv sedan ett program i Python som beräknar (5)
med den (i \emph{SciPy}) inbyggda
\end{quote}
\end{minipage}} \\
\bottomrule()
\end{longtable}

\begin{longtable}[]{@{}
  >{\raggedright\arraybackslash}p{(\columnwidth - 4\tabcolsep) * \real{0.3333}}
  >{\raggedright\arraybackslash}p{(\columnwidth - 4\tabcolsep) * \real{0.3333}}
  >{\raggedright\arraybackslash}p{(\columnwidth - 4\tabcolsep) * \real{0.3333}}@{}}
\toprule()
\begin{minipage}[b]{\linewidth}\raggedright
\end{minipage} & \begin{minipage}[b]{\linewidth}\raggedright
\end{minipage} & \begin{minipage}[b]{\linewidth}\raggedright
3 (4)
\end{minipage} \\
\midrule()
\endhead
\begin{minipage}[t]{\linewidth}\raggedright
\begin{quote}
\includegraphics[width=0.63889in,height=0.63889in]{vertopal_35c4d4bda5e64349bff9adbb02318ace/media/image6.png}
\end{quote}
\end{minipage} & \multirow{4}{*}{1} &
\multirow{3}{*}{\begin{minipage}[t]{\linewidth}\raggedright
integral lösaren quad och redovisa resultatet. Då kurvan för
\textbar{}\emph{𝑣}(\emph{𝑡})\textbar{} är rela-tivt osnäll så är det
lämpligt att höja noggrannheten i beräkningen. I python

kan anropet till quad se ut som följande (där vi skapat en funktion som

\begin{quote}
heter \emph{vel}, och där vi tilldelat värden för alla parametrar),
\end{quote}
\end{minipage}} \\
\begin{minipage}[t]{\linewidth}\raggedright
\begin{quote}
\includegraphics[width=0.80556in,height=0.15278in]{vertopal_35c4d4bda5e64349bff9adbb02318ace/media/image7.png}
\end{quote}
\end{minipage} \\
\multirow{2}{*}{\includegraphics[width=1.16667in,height=0.13889in]{vertopal_35c4d4bda5e64349bff9adbb02318ace/media/image8.png}} \\
& & \begin{minipage}[t]{\linewidth}\raggedright
\begin{quote}
from scipy.integrate import quad, trapezoid
\end{quote}
\end{minipage} \\
\bottomrule()
\end{longtable}

2

\begin{quote}
3 S, err = quad(vel, 0, T, args = (omega0,omega,gamma), ...
\end{quote}

epsabs=1.49e-12, epsrel=1.49e-12, limit=500)

\begin{quote}
Här kan man fråga sig om man kommer se förväntad konvergens med Simpsons
formel. Notera att strikta felformeln för Simpson innehåller en
fjärdederivata av integranden, dvs absolutbeloppet av vinkelhastigheten
given av (4). Ser ni också någon varning då ni anropar quad i detta
fall?

Man ser att hastigheten för första gången (efter t=0) blir noll
nära\emph{𝑡}= 0\emph{.}255. Testa nu att räkna (med quad) mellan
\emph{𝑡}= 0 och \emph{𝑡}= 0\emph{.}25, där lösningen är snäll i hela
intervallet, där ni också plottar upp
\textbar{}\emph{𝑣}(\emph{𝑡})\textbar. Använd\emph{𝑁}= 200 intervall i
detta fall.
\end{quote}

\textbf{Deluppgift 2: \emph{Konvergens}}

\begin{longtable}[]{@{}
  >{\raggedright\arraybackslash}p{(\columnwidth - 6\tabcolsep) * \real{0.2500}}
  >{\raggedright\arraybackslash}p{(\columnwidth - 6\tabcolsep) * \real{0.2500}}
  >{\raggedright\arraybackslash}p{(\columnwidth - 6\tabcolsep) * \real{0.2500}}
  >{\raggedright\arraybackslash}p{(\columnwidth - 6\tabcolsep) * \real{0.2500}}@{}}
\toprule()
\multicolumn{4}{@{}>{\raggedright\arraybackslash}p{(\columnwidth - 6\tabcolsep) * \real{1.0000} + 6\tabcolsep}@{}}{%
\begin{minipage}[b]{\linewidth}\raggedright
\begin{quote}
Skriv ett program i Python som löser (5) med trapetsformeln, genom att
anropa funktionen trapezoid. Sätt \emph{𝜔}= 2 \emph{𝜋}, \emph{𝜔}0 = 3
\emph{𝜋}, \emph{𝛾}= 0\emph{.}1. Sätt begynnelsevillkoren till \emph{𝜃}=
\emph{𝜃}′= 0. Sluttiden sätts till \emph{𝑇}= 10. Låt \emph{𝑁}beteckna
antalet steg (intervall). Tidssteget (\emph{𝑘}) ges då enligt
\emph{𝑘}=\emph{𝑇𝑁}. Beräkna (5) med trapetsformen med \emph{𝑁}=
100\emph{,} 200\emph{,} 400\emph{,} 800 intervall och redovisa
resultaten i en tabell (se Tabell 1, för tydligt redovisad tabell-mall).
Låt\emph{𝑆𝑇}(\emph{𝑁}) beteckna trapetsformeln med \emph{𝑁}intervall.
Använd nu 3e-dels regeln för att uppskatta felen (som vi betecknar med
\emph{𝑒𝑁}) hos \emph{𝑆𝑇}(200), \emph{𝑆𝑇}(400) och

tabell. Redovisa därefter den uppskattade konvergensen med hjälp av
(6).\emph{𝑆𝑇}(800). Redovisa de uppskattade felen, dvs \emph{𝑒}200,
\emph{𝑒}400 och \emph{𝑒}800 i samma

Man bör här se 2a ordningens konvergens (om funktionen är tillräckligt
snäll), dvs att felet skalar som ≃\emph{𝑁}−2. Om man mäter felen med
\emph{𝑁}och 2\emph{𝑁}steg kan man uppskatta metodens
noggrannhets-ordningen (konvergensen), som vi betecknar med
\emph{𝑞}(2\emph{𝑁}), med följande formel:
\end{quote}
\end{minipage}} \\
\midrule()
\endhead
\multirow{2}{*}{\emph{𝑞}(2\emph{𝑁}) = \emph{𝑙𝑜𝑔}2} & \emph{𝑒𝑁} &
\begin{minipage}[t]{\linewidth}\raggedright
\begin{quote}
\emph{.}
\end{quote}
\end{minipage} & \multirow{2}{*}{(6)} \\
& \emph{𝑒}2\emph{𝑁} & \\
\multicolumn{4}{@{}>{\raggedright\arraybackslash}p{(\columnwidth - 6\tabcolsep) * \real{1.0000} + 6\tabcolsep}@{}}{%
\begin{minipage}[t]{\linewidth}\raggedright
\begin{quote}
Beräkna nu sträckan med Simpsons formel som vi här betecknar som

kan räknas fram med hjälp av de tidigare trapetsberäkningarna och
de\emph{𝑆𝑆}(\emph{𝑁}), då \emph{𝑁}= 200, \emph{𝑁}= 400 och \emph{𝑁}=
800. Notera att dessa värden

uppskattade felen. Redovisa resultaten i samma tabell. Använd nu 15-dels
regeln för att uppskatta felen hos \emph{𝑆𝑆}(400) och \emph{𝑆𝑆}(800),
som ni redovisar i samma tabell. Redovisa till sist den uppskattade
konvergensen, där ni jämför felen för olika steglängder. Man bör här se
4de ordningens konvergens om funktionen är tillräckligt snäll, dvs att
felet skalar som ≃\emph{𝑁}−4. Frågan är
\end{quote}
\end{minipage}} \\
\bottomrule()
\end{longtable}

\begin{longtable}[]{@{}
  >{\raggedright\arraybackslash}p{(\columnwidth - 14\tabcolsep) * \real{0.1250}}
  >{\raggedright\arraybackslash}p{(\columnwidth - 14\tabcolsep) * \real{0.1250}}
  >{\raggedright\arraybackslash}p{(\columnwidth - 14\tabcolsep) * \real{0.1250}}
  >{\raggedright\arraybackslash}p{(\columnwidth - 14\tabcolsep) * \real{0.1250}}
  >{\raggedright\arraybackslash}p{(\columnwidth - 14\tabcolsep) * \real{0.1250}}
  >{\raggedright\arraybackslash}p{(\columnwidth - 14\tabcolsep) * \real{0.1250}}
  >{\raggedright\arraybackslash}p{(\columnwidth - 14\tabcolsep) * \real{0.1250}}
  >{\raggedright\arraybackslash}p{(\columnwidth - 14\tabcolsep) * \real{0.1250}}@{}}
\toprule()
\begin{minipage}[b]{\linewidth}\raggedright
\end{minipage} &
\multicolumn{7}{>{\raggedright\arraybackslash}p{(\columnwidth - 14\tabcolsep) * \real{0.8750} + 12\tabcolsep}@{}}{%
\begin{minipage}[b]{\linewidth}\raggedright
4 (4)
\end{minipage}} \\
\midrule()
\endhead
\multirow{3}{*}{\begin{minipage}[t]{\linewidth}\raggedright
\begin{quote}
\includegraphics[width=0.63889in,height=0.63889in]{vertopal_35c4d4bda5e64349bff9adbb02318ace/media/image6.png}

\includegraphics[width=0.80556in,height=0.15278in]{vertopal_35c4d4bda5e64349bff9adbb02318ace/media/image7.png}
\end{quote}

\includegraphics[width=1.16667in,height=0.13889in]{vertopal_35c4d4bda5e64349bff9adbb02318ace/media/image8.png}
\end{minipage}} &
\multicolumn{7}{>{\raggedright\arraybackslash}p{(\columnwidth - 14\tabcolsep) * \real{0.8750} + 12\tabcolsep}@{}}{%
\begin{minipage}[t]{\linewidth}\raggedright
\begin{quote}
här om det antagandet gäller, då funktionen som ges av absolutbeloppet
av (4) inte är speciellt snäll nära vändpunkterna (då hastigheten är
noll).
\end{quote}
\end{minipage}} \\
& \emph{𝑁} & \multirow{2}{*}{\emph{𝑆𝑇}(\emph{𝑁})} &
\multirow{2}{*}{\emph{𝑒𝑁}} & \multirow{2}{*}{q(N)} &
\multirow{2}{*}{\emph{𝑆𝑆}(\emph{𝑁})} & \multirow{2}{*}{\emph{𝑒𝑁}} &
\multirow{2}{*}{\begin{minipage}[t]{\linewidth}\raggedright
\begin{quote}
q(N)
\end{quote}
\end{minipage}} \\
& \multirow{2}{*}{100} \\
\multirow{6}{*}{} & &
\multirow{5}{*}{\begin{minipage}[t]{\linewidth}\raggedright
\begin{quote}
\emph{𝑆𝑇}(100)\emph{𝑆𝑇}(200)\emph{𝑆𝑇}(400)\emph{𝑆𝑇}(800)
\end{quote}
\end{minipage}} & \multirow{3}{*}{\emph{𝑒}200} &
\multirow{5}{*}{\emph{𝑞}(400)\emph{𝑞}(800)} & &
\multirow{4}{*}{\emph{𝑒}400} &
\multirow{5}{*}{\begin{minipage}[t]{\linewidth}\raggedright
\begin{quote}
\emph{𝑞}(800)
\end{quote}
\end{minipage}} \\
& 200 & & & &
\multirow{4}{*}{\begin{minipage}[t]{\linewidth}\raggedright
\begin{quote}
\emph{𝑆𝑆}(200)\emph{𝑆𝑆}(400)\emph{𝑆𝑆}(800)
\end{quote}
\end{minipage}} \\
& \multirow{2}{*}{400} \\
& & & \emph{𝑒}400 \\
& 800 & & \emph{𝑒}800 & & & \emph{𝑒}800 \\
&
\multicolumn{7}{>{\raggedright\arraybackslash}p{(\columnwidth - 14\tabcolsep) * \real{0.8750} + 12\tabcolsep}@{}}{%
\begin{minipage}[t]{\linewidth}\raggedright
\begin{quote}
Tabell 1: Mall för en tydlig tabell över resultat, feluppskattning och
konver-gens. Ersätt med beräknade resultaten.
\end{quote}
\end{minipage}} \\
\bottomrule()
\end{longtable}

\begin{quote}
Vi ska nu genomföra samma konvergens verifiering, men med skillna-den
att vi räknar till sluttiden \emph{𝑇}= 0\emph{.}25, istället som
tidigare \emph{𝑇}= 10. I detta fall gäller antagandet om en snäll
lösning i hela intervallet. Använd i övrigt samma parametrar, och
redovisa resultatet i form av Tabell 1. Får ni nu förväntad konvergens
även för Simpsons formel?

Som en extra kontroll av de uppskattade felen kan ni jämföra skillna-den
mellan de framräknade resultaten och det värde ni fick med quad i
deluppgift 1.
\end{quote}

\textbf{Deluppgift 3: \emph{Kaotisk pendel}}

\begin{quote}
Utgå från ert tidigare program från projekt 1 (deluppgift 4), som löser
(2) med ODE lösaren solve\_ivp, med begynnelsevillkoren satta
till\emph{𝜃}= \emph{𝜃}′= 0 och där parametrarna sätts till \emph{𝜔}= 2
\emph{𝜋}, \emph{𝜔}0 =\uline{3 2}\emph{𝜔}, \emph{𝛽}=\sout{1 4}\emph{𝜔}0
och\emph{𝛾}= 1\emph{.}07. Räkna fram till sluttiden \emph{𝑇}= 40. Använd
nu lösningen (dvs \emph{𝑡}och
\end{quote}

anropa funktionen trapezoid. Redovisa resultatet.\emph{𝜃}′(\emph{𝑡}))
från solve\_ivp för att beräkna (5) med trapetsformeln, genom att

\begin{quote}
Vilka olika trunkeringsfel inför vi i beräkningen av sträckan och hur
kan man uppskatta dessa? (Ni behöver här inte genomföra några
beräkningar, endast argumentera och resonera.)
\end{quote}

\textbf{Valfri uppgift: \emph{Konvergens för ickelinjär model}}

\begin{quote}
Utgå från ert tidigare program från projekt 1 (valfri uppgift), som
löser (2) med RK4, med begynnelsevillkoren satta till \emph{𝜃}=
\emph{𝜃}′= 0 och där parametrarna sätts till \emph{𝜔}= 2 \emph{𝜋},
\emph{𝜔}0 =\uline{3 2}\emph{𝜔}, \emph{𝛽}=\sout{1 4}\emph{𝜔}och \emph{𝛾}=
1\emph{.}07. Räkna fram till sluttiden \emph{𝑇}= 40. Genomför
beräkningen med \emph{𝑁}= 8000.

Använd nu lösningen (dvs \emph{𝑡}och \emph{𝜃}′(\emph{𝑡})) från RK4
beräkningen för att beräkna (5) med trapetsformen, som vi betecknar
\emph{𝑆𝑇}(8000). Beräkna sedan
\end{quote}

punkt i lösningsvektorerna. Redovisa resultaten i en tabell. Använd nu
3e-\emph{𝑆𝑇}(4000) och \emph{𝑆𝑇}(2000) genom att använda varannan
respektive var fjärde

\begin{quote}
dels regeln för att uppskatta felen hos \emph{𝑆𝑇}(8000) och
\emph{𝑆𝑇}(4000), dvs \emph{𝑒}8000 och \emph{𝑒}4000 och redovisa i samma
tabell. Redovisa därefter den uppskattade konvergensen med (6). Redovisa
till sist resultatet med Simpsons metod då\emph{𝑁}= 8000. Försök till
sist uppskatta konvergensen med Simpsons formel.
\end{quote}

\begin{longtable}[]{@{}
  >{\raggedright\arraybackslash}p{(\columnwidth - 2\tabcolsep) * \real{0.5000}}
  >{\raggedright\arraybackslash}p{(\columnwidth - 2\tabcolsep) * \real{0.5000}}@{}}
\toprule()
\begin{minipage}[b]{\linewidth}\raggedright
\end{minipage} & \begin{minipage}[b]{\linewidth}\raggedright
5 (4)
\end{minipage} \\
\midrule()
\endhead
\begin{minipage}[t]{\linewidth}\raggedright
\begin{quote}
\includegraphics[width=0.63889in,height=0.63889in]{vertopal_35c4d4bda5e64349bff9adbb02318ace/media/image6.png}

\includegraphics[width=0.80556in,height=0.15278in]{vertopal_35c4d4bda5e64349bff9adbb02318ace/media/image7.png}
\end{quote}

\includegraphics[width=1.16667in,height=0.13889in]{vertopal_35c4d4bda5e64349bff9adbb02318ace/media/image8.png}
\end{minipage} & \begin{minipage}[t]{\linewidth}\raggedright
\begin{quote}
Får ni för detta fall den förväntade 4de ordningens konvergens (som
bygger på antagandet att funktionen är tillräckligt snäll)?

Vilka olika trunkeringsfel inför vi i beräkningen av sträckan och hur
kan man uppskatta dessa? Hur kan man försäkra sig om att
trunkeringsfelet i trapetsberäkningen dominerar?
\end{quote}
\end{minipage} \\
\bottomrule()
\end{longtable}

\end{document}
